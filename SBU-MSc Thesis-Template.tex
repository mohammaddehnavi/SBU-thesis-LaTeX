% SBU-Thesis v1.0

% این قالب بر اساس فرمت پایان‌نامه‌ها و رساله‌های تحصیلات تکمیلی دانشگاه اصفهان تهیه شده است.
% علیرضا روحی-دانشجوی دکتری گروه مهندسی نرم افزار دانشگاه اصفهان
% 1395
% rouhi.ir@gmail.com
% توصیه می‌شود که از توزیع تک‌لایو (TexLive2015) به بعد استفاده شود:
% http://tug.org/texlive/acquire-iso.html

% موفق باشید.
% با تشکر از امین فخاری که قالب اصلی این پایان نامه را برای دانشگاه صنعتی اصفهان تهیه نموده اند.
% 1395
% a101.fakhari@gmail.com
% -----------------------------------------------------------------------------------

% نکات:

% برای آن‌که پردازش فایل و مشاهده خروجی در هنگام نوشتن پایان‌نامه آسان‌تر و سریع‌تر انجام شود، انجام موارد زیر توصیه می گردد:
% الف) فصل‌ها و بخش‌هایی که در حال نوشتن آن‌ها نیستید را غیر فعال کنید. به‌عنوان مثال، در این قالب، این دستورات را می‌توان در صورت عدم نیاز با اضافه کردن % به طور موقت غیرفعال کرد:
% \MakeTitlePage
% \MakeFarsiSignaturePage
% \input{Chapters/Acknowledgments}
% \MakeCopyRightPage
% \input{Chapters/Dedication}
% \MakeTableOfContents
% \MakeListOfFigures
% \MakeListOfTables
% \MakeFarsiAbstract
% \input{Chapters/Chapter#}
% \MakeAppendices
% \input{Chapters/Appendices}
% \MakeEnglishAbstract
% \MakeEnglishSignaturePage
% ب) از گزینه draft برای فراخوانی کلاس استفاده کنید. یعنی
% \documentclass[a4paper,fleqn,13pt,twoside,draft]{book}
% این گزینه حالت چرکنویس را ایفا می‌کند و بر روی بسته‌های مختلف اثرهای متفاوتی دارد. به‌عنوان مثال: به جای شکل، تنها چهارچوب آن نمایش داده شود، لینک‌های hyperref غیر فعال گردد، فایل‌های خارجی را در بسته listings اضافه نمی‌کند و ... و همه این موارد سبب کاهش زمان اجرا و حجم فایل می‌شود.

% در صورتی که میخواهید به سطر بعد بروید اما نمیخواهید بین دو کلمه‌ای که نوشتید فاصله بیفتد کافی است در انتهای خط اول  (بدون فاصله) کاراکتر % را اضافه کنید. با این عمل، لاتک خط فاصله ایجاد شده در اثر تغییر سطر را به عنوان توضیح اضافه یا کامنت در نظر میگیرد و در خروجی اعمال نمی‌کند.

% توصیه می‌شود از شکل‌های برداری با فرمت PDF استفاده شود. این کار علاوه بر افزایش کیفیت رسال/پایان‌نامه/گزارش، باعث کاهش حجم شکل‌ها (و در نتیجه  کاهش حجم فایل نهایی) و همچنین کاهش زمان پردازش می‌شود.

% در این قالب سعی شده است که از تمامی بخش‌های موجود در پایان‌نامه‌ها نمونه‌ای آورده شود.

\documentclass[a4paper,fleqn,11pt,twoside]{book}
\usepackage{Settings/SBU-Thesis}
\usepackage[nolist]{acronym}

%-----------------------------
% دستورهای مورد نیاز را در این قسمت اضافه نمایید:
% Cross-reference commands.
\newtheorem{thm}{Theorem}[chapter]
\theoremstyle{definition}
\newtheorem{defn}[thm]{تعریف}

\newcommand{\xs}[1]{بخش~\ref{#1}}
\newcommand{\xc}[1]{فصل~\ref{#1}}
\newcommand{\xp}[1]{صفحه~\pageref{#1}}
\newcommand{\xf}[1]{شکل~\ref{#1}}
\newcommand{\xt}[1]{جدول~\ref{#1}}
\newcommand{\xa}[1]{پیوست~\ref{#1}}
\newcommand{\xd}[1]{تعریف~\ref{#1}}
\newcommand{\xr}[1]{قانون~\ref{#1}}
\newcommand{\xra}[1]{R~\ref{#1}}
\newcommand{\xl}[1]{کد~\ref{#1}}
\newcommand{\xal}[1]{الگوریتم~\ref{#1}}


\makeatletter
\providecommand*{\cupdot}{%
  \mathbin{%
    \mathpalette\@cupdot{}%
  }%
}
\newcommand*{\@cupdot}[2]{%
  \ooalign{%
    $\m@th#1\cup$\cr
    \hidewidth$\m@th#1\cdot$\hidewidth
  }%
}
\makeatother

%\DeclareMathSizes{9}{9}{9}{9}

\lstset{escapeinside={/*@}{@*/}}
%\definecolor{codebackground}{rgb}{0.95,0.95,0.95}
\definecolor{codebackground}{RGB}{255,255,255}
\definecolor{commentcolor}{RGB}{77,153,153}
\definecolor{keywordcolor}{RGB}{153,77,153}
\lstset{backgroundcolor=\color{codebackground}}

\lstset{
  captionpos=b,
  numberstyle=\tiny,
  %basicstyle=\ttfamily\footnotesize,
  basicstyle=\setLTR\footnotesize\ttfamily,
  columns=flexible,
  tabsize=2,
  numbers=none, %left,
  nolol=true,
  keywordstyle=\color{keywordcolor},
  commentstyle=\color{commentcolor},
  stringstyle=\color{blue},
  captiondirection=RTL,
  upquote=true,
}

\def\lstlistingname{کد}

%-----------------------------

\begin{document}

\pagestyle{plain}
\pagenumbering{harfi}
\setcounter{page}{2}

% ░░░░░░░▒▒▒▒▒▒▓▓▓▓ In the Name of Allah ▓▓▓▓▒▒▒▒▒▒░░░░░░░
\clearpage
\thispagestyle{empty}
\newgeometry{left=3.5cm, right=3.5cm, top=3.5cm, bottom=3.5cm}
\begin{figure}
  \centering
\includegraphics[width = \linewidth]{Settings/Allah.png}
\end{figure}
% ░░░░░░░▒▒▒▒▒▒▓▓▓▓ Title Page ▓▓▓▓▒▒▒▒▒▒░░░░░░░
\DepartmentFa{دانشکده مهندسی و علوم کامپیوتر}
%\GroupFa{گروه مهندسی نرم‌افزار}
\ThesisTypeFa{پایان‌نامه} % Or \ThesisTypeFa{رساله} Or \ThesisTypeFa{پیشنهادیه پایان‌نامه}
\DegreeFa{کارشناسی ارشد} % Or \DegreeFa{دکتری} 
\ThesisMark{عالی} % Or \ThesisMark{بسیارخوب} Or \ThesisMark{خوب}
\FieldFa{مهندسی کامپیوتر}
\BranchFa{معماری سیستم‌های کامپیوتری}
\YourFullnameFa{نام و نام خانوادگی دانشجو}
\FirstSupervisorFa{نام و نام خانوادگی استاد راهنما}
\SecondSupervisorFa{نام و نام خانوادگی استاد راهنمای دوم}
% Optional (Remove It If You Don't Have)
\YearFa{پاییز 1401}
\TitleFa{عنوان فارسی پایان نامه}

% اگر عنوان رساله طولانی بود، در دو خط به صورت نشان داده شده تقسیم شود.
%\TitleFa{قسمت اول عنوان \\ [0.4cm] قسمت دوم عنوان}

\MakeTitlePage


% ░░░░░░░▒▒▒▒▒▒▓▓▓▓ Signature - Farsi ▓▓▓▓▒▒▒▒▒▒░░░░░░░
%\Prefix{آقای} %\Prefix{خانم}
\DateFa{1401/11/11}
%\FirstSupervisorAcademicRank{استادیار} % Or \FirstSupervisorAcademicRank{دانشیار} Or \FirstSupervisorAcademicRank{استاد}
%\SecondSupervisorAcademicRank{استادیار} % Or \SecondSupervisorAcademicRank{دانشیار} Or \SecondSupervisorAcademicRank{استاد}
\FirstAdvisorFa{مشاور اول}
%\FirstAdvisorAcademicRank{استادیار} % Or \FirstAdvisorAcademicRank{دانشیار} Or \FirstAdvisorAcademicRank{استاد}
%\SecondAdvisorFa{مشاور دوم} % Optional (Remove It If You Don't Have)
%\SecondAdvisorAcademicRank{استادیار} % Or \SecondAdvisorAcademicRank{دانشیار} Or \SecondAdvisorAcademicRank{استاد}
%\FirstExaminerFa{نام و نام خانوادگی}
%\FirstExaminerAcademicRank{استادیار} % Or \FirstExaminerAcademicRank{دانشیار} Or \FirstExaminerAcademicRank{استاد}
%\SecondExaminerFa{نام و نام خانوادگی} % Optional (Remove It If You Don't Have)
%\SecondExaminerAcademicRank{استادیار} % Or \SecondExaminerAcademicRank{دانشیار} Or \SecondExaminerAcademicRank{استاد}
%\ThirdExaminerFa{نام و نام خانوادگی} % Optional (Remove It If You Don't Have)
%\ThirdExaminerAcademicRank{استادیار} % Or \SecondExaminerAcademicRank{دانشیار} Or \SecondExaminerAcademicRank{استاد}
%\DeanOfDepartmentFa{دکتر تحصیلات تکمیلی دانشکده}

\MakeFarsiSignaturePage

% ░░░░░░░▒▒▒▒▒▒▓▓▓▓ CopyRight ▓▓▓▓▒▒▒▒▒▒░░░░░░░
\MakeCopyRightPage
\MakeCopyRightPageTwo

% ░░░░░░░▒▒▒▒▒▒▓▓▓▓ Acknowledgments ▓▓▓▓▒▒▒▒▒▒░░░░░░░
\input{Chapters/Acknowledgments}

% ░░░░░░░▒▒▒▒▒▒▓▓▓▓ Dedication ▓▓▓▓▒▒▒▒▒▒░░░░░░░
\input{Chapters/Dedication}

% ░░░░░░░▒▒▒▒▒▒▓▓▓▓ Abstract - Farsi ▓▓▓▓▒▒▒▒▒▒░░░░░░░
\input{Chapters/AbstractFa}
\MakeFarsiAbstract

% ░░░░░░░▒▒▒▒▒▒▓▓▓▓ Table of Contents/Figures/Tables ▓▓▓▓▒▒▒▒▒▒░░░░░░░
\MakeTableOfContents
\MakeListOfFigures
\MakeListOfTables

% ----------------------------------------------------------------------------
\clearpage
\pagestyle{myheadings}
\pagenumbering{arabic}
\setcounter{page}{1}

% ░░░░░░░▒▒▒▒▒▒▓▓▓▓ Added by mohi for show chapter name in new page ▓▓▓▓▒▒▒▒▒▒░░░░░░░
\titleformat{\chapter}[display]
{\vspace{-3cm}\vfill\filcenter}
{{%
		\vspace{-3cm}\filcenter\fontsize{40pt}{40pt}\selectfont{\chaptername}
		\fontsize{40pt}{40pt}\selectfont\tartibi{chapter}
	}%
}
{50pt}
{\fontsize{30pt}{30pt}\selectfont%
}[\vfill\clearpage]
\titlespacing*{\chapter}{0pt}{0pt}{0pt}
%

% ░░░░░░░▒▒▒▒▒▒▓▓▓▓ Chapters ▓▓▓▓▒▒▒▒▒▒░░░░░░░
\clearpage
\baselineskip=0.9cm

% ░░░░░░░▒▒▒▒▒▒▓▓▓▓ Added by mohi for show chapter name in header ▓▓▓▓▒▒▒▒▒▒░░░░░░░
\pagestyle{fancy}
\renewcommand{\chaptermark}[1]{\markboth{#1}{#1}}
\fancyhead[R]{}
\fancyhead[L]{\chaptername\ \tartibi{chapter} : \leftmark}
%

\input{Chapters/Chapter1}
\input{Chapters/Chapter2}
\input{Chapters/Chapter3}

% ░░░░░░░▒▒▒▒▒▒▓▓▓▓ Added by mohi for remove fancy header ▓▓▓▓▒▒▒▒▒▒░░░░░░░
\pagestyle{plain}

% ░░░░░░░▒▒▒▒▒▒▓▓▓▓ References ▓▓▓▓▒▒▒▒▒▒░░░░░░░
\MakeReferences
\bibliographystyle{Settings/ModifiedIEEEtranFa}
\bibliography{References}

% ░░░░░░░▒▒▒▒▒▒▓▓▓▓ Appendices ▓▓▓▓▒▒▒▒▒▒░░░░░░░
\MakeAppendices
\input{Chapters/Appendices}

% ░░░░░░░▒▒▒▒▒▒▓▓▓▓ Abstract - English ▓▓▓▓▒▒▒▒▒▒░░░░░░░
\input{Chapters/AbstractEn}


% ░░░░░░░▒▒▒▒▒▒▓▓▓▓ Signature - English ▓▓▓▓▒▒▒▒▒▒░░░░░░░
\DepartmentEn{Faculty of Computer Science and Engineering}
\DegreeEn{M.Sc.} % Or \DegreeEn{Ph.D.}
\YourFullnameEn{Student First and Last Name}
\DateEn{December 2022}
\FirstSupervisorEn{Supervisor First and Last Name}
%\SecondSupervisorEn{Second Supervisor, Prof.} % Optional (Remove It If You Don't Have)
\TitleEn{Thesis English Title \\[0.2cm]}
% اگر عنوان رساله طولانی بود، در دو خط به صورت نشان داده شده تقسیم شود.
\GroupEn{Department of Software Engineering}
\FirstAdvisorEn{First Advisor, Assoc. Prof.}


\MakeEnglishAbstract
\MakeEnglishSignaturePage

\end{document} 

